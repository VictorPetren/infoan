\documentclass[a4paper]{article}
\usepackage[utf8]{inputenc}
\usepackage{amsmath}
\usepackage{amssymb}
\usepackage{mathtools}
\usepackage{amsfonts}
\usepackage{lastpage}
\usepackage{pdfpages}
\usepackage{fancyvrb}
\usepackage[table]{colortbl}
\usepackage{fancyhdr}
\usepackage[linesnumbered, ruled]{algorithm2e}
\usepackage{graphicx}
\SetKwRepeat{Do}{do}{while}%
\usepackage[margin=2.5 cm]{geometry}


\pagestyle{fancy}
\cfoot{Page \thepage\ of \pageref{LastPage}}
\DeclareGraphicsExtensions{.pdf,.png,.jpg}
\author{Sigurdur Oli Arnason (5961181) \\ Victor Petren Bach Hansen (5990025)}
\title{Algorithms \& Networks \\ Exercise 3}
\lhead{Algorithms \& Networks}
\rhead{Exercise 2}

\begin{document}
\maketitle
\section{Counting numbers}
For $N=10000$, let $|P_k|$ denote the number of integers in the range $[1,2,\ldots,N]$ that $k$ divides. This number is equal to $\lfloor \frac{N}{k}\rfloor $. Let $|P_k \cap P_l|$ denote the number of integers in the range $[1,2,\ldots,N]$ that both $k$ and $l$ divides. This is equal to $\lfloor \frac{N}{k*l}\rfloor $. Using the inclusion/exclusion principle, we get that the number of integers, $D$, in the range $[1,2,\ldots, N]$ that is divisible by either 5, 11 or 19, is:
\begin{align*}
  D &=|P_5|+|P_{11}| + |P_{19}|-|P_{5}\cap P_{11}|-|P_{5}\cap P_{19}|-|P_{11}\cap P_{19}|+|P_{5}\cap P_{11} \cap P_{19} |\\
    &= 2000+909+564-181-105-47+9\\
    &= 3111
\end{align*}

\section{Inclusion/Exclusion for counting perfect matchings}
\subsection*{i)}
Since we in a perfect matching only have $n/2$ edges to cover all vertices $v\in V$, in order to satisfy the property that every $v$ must be incident to \textit{at least one} edge in $M$, each edge must be used to cover 2 unmatched vertices. If we don't it would mean that we e.g. used two edges to cover 3 vertices, resulting in 1 vertex being unmatched. Therefore, the number of perfect matchings is equal to the different number of ways we can chose a matching of size $n/2$, where each vertex has
to be incident to at least one edge.
\subsection*{ii)}
If we define the, for some $U \subseteq V $, the induced subgraph $G[U]=G'$, where $G'=(U,F)$, then this corresponds to the number of ways we can select $n/2$ different edges from $F$. We will call this number $N$ and the it has the formula
$$
  N=\binom{|F|}{n/2}
$$
This can easily be computed in polynomial time.
\subsection*{iii)}
We want to count the number of perfect matching some graph $G$ has. We know from $i)$ that this number is equal to the number of different ways we can arrange $n/2$ such that it satisfies the property that all vertices in $V$ are incident to at least one edge in the matching $M$. Surely, if we compute all the different ways we can choose $n/2$ edges from $G$, all perfect matches should be contained within these. However since we counted to many, we need to remove all non-perfect matches. If
we remove a vertex $v$ from $G$, ie $G[V\setminus \{v\}]$, the number of ways we can choose $n/2$ edges would all not be perfect matches, as this accounts for all the cases where $v$ is unmatched, so we can remove these. If we do this for each $v\in V$, we will count some cases twice (the cases where 2 vertices are unmatched), so we will have to add these again by counting the number of ways we can chose $n/2$ edges when removing all subsets of size 2. Continuing this approach, yields the
following inclusion/Exclusion formula:
$$
\sum_{X\subseteq V} (-1)^{|X|} N(G[V \setminus X])
$$
where the $X$ is all possible subsets of $V$ (incl the empty set) and the function $N$ is the result from $ii)$:
$$
N(G=(F,U)) = \binom{|F|}{n/2}
$$
\section{Sort and Search for exact satisfiability }
\section{Maximum value matching and minimum cost flow}
As with the maximum matching in a bipartite graph approach, we transform $G$ into $G'$ by adding a source $s$ and a sink $t$ to the vertex set, and connect $s$ to all vertices in $V_1$ and $t$ to all vertices in $V_2$. We then assign a capacity of $1$ to all edges in the new graph $G'$. We also assign a cost to all outgoing edges of $s$ to $0$ and the same for all incoming edges of $t$. For all the edges $e$ that connects $V_1$ and $V_2$, we assign a cost of $-w(e)$.

Now the problem can be solved by finding a minimum cost flow from $s$ to $t$, which can be done using e.g. the cycle cancelling algorithms described in the lectures.
\section{Maximum size simple b-matching}
\subsection*{1}
\subsection*{2}
\subsection*{3}
\subsection*{4}
\subsection*{5}
\subsection*{6}

\end{document}

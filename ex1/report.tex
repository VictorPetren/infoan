\documentclass[a4paper]{article}
\usepackage[utf8]{inputenc}
\usepackage{amsmath}
\usepackage{amssymb}
\usepackage{mathtools}
\usepackage{amsfonts}
\usepackage{lastpage}
\usepackage{pdfpages}
\usepackage{fancyvrb}
\usepackage[table]{colortbl}
\usepackage{fancyhdr}
\usepackage[linesnumbered, ruled]{algorithm2e}
\SetKwRepeat{Do}{do}{while}%
\usepackage[margin=2.5 cm]{geometry}

\pagestyle{fancy}
\cfoot{Page \thepage\ of \pageref{LastPage}}
\DeclareGraphicsExtensions{.pdf,.png,.jpg}
\author{Authors}
\title{Algorithms \& Networks \\ Exercise 1}
\lhead{Algorithms \& Networks}
\rhead{Exercise 1}

\begin{document}
\maketitle
\section{Strongly connected components and strongly connected graphs}
\subsection*{i}
Since $v$ and $w$ are mutually reachable then we have a path from $v$ to $w$ and a path from $w$ to $v$. Since $w$ and $x$ are also mutually reachable then we have a path from $w$ to $x$ and a path from $x$ to $w$.\\
For $v$ and $x$ to be mutually reachable we need:
\begin{enumerate}
  \item A path from $v$ to $x$
  \item A path from $x$ to $v$
\end{enumerate}
We get (1) by using the path from $v$ to $w$ and then the path from $w$ to $x$ and we get (2) by using the path from $x$ to $w$ and then the path from $w$ to $v$. Therefore $x$ and $v$ are mutually reachable.

\subsection*{ii}
Since $X$ is the strongly connected component of $v$ there is a path from $v$ to $w$ and also a path from $w$ to $v$ for all $w \in X$. That is, $v$ and $w$ are mutually reachable for all $w \in X$. By $(i)$ we then have that all pairs of vertices in $X$ are mutually reachable so for all $w \in X$, $X$ is at least a subset of the strongly connected component for $w$.

If there was a vertex that was in the strongly connected component of $w$ but not in $X$ then there would be a path from $u$ to $v$ and $v$ to $u$ and by $(i)$ $u$ would then be in $X$ which is a contradiction. \\
That means that $X$ is exactly the strongly connected component of $w$.
\subsection*{iii}
In order to compute the set $\{w \in V | \mbox{ there is a path from $v$ to $w$ in $G$}\}$, for a given $v$ in $G$, we propose the following algorithm:
\begin{enumerate}
  \item Keep a list of visited vertices.
  \item From $v$, visit all unvisited vertices, $u$, if there is an edge $(v,u)$, and mark them as visited.
  \item Repeat for newly visited vertices recursively until stops.
  \item The desired set is the list of visited vertices
\end{enumerate}
This algorithm uses each edge and vertex at most once so it is $O(n + m)$.

\subsection*{iv}
In order to compute the set $\{w \in V | \mbox{ there is a path from $w$ to $v$ in $G$}\}$, for a given $v$ in $G$, we propose the following algorithm:
\begin{enumerate}
  \item Reverse the direction of all edges in $A$
  \item Run the algorithm described in $(iii)$
\end{enumerate}
Reversing the edges takes $O(m)$ time and running the algorithm in $(iii)$ takes $O(n+m)$ time, therefore the total cost is $O(n+m)$.
\subsection*{v}
For any vertex $v \in V$ run both the algorithms described in $(iii)$ and $(iv)$. If they both return the set $V$, i.e. all vertices of $G$, then the graph is strongly connected, otherwise not.
\subsection*{vi}
For a DAG you never have, for a pair of vertices $v$ and $u$, both a path from $v$ to $u$ and a path from $u$ to $v$ as this would imply a cycle, so the strongly connected components are the vertices themselves, assuming that a vertex is strongly connected to itself.

\section{A stable marriage for Eric and Ariel?}
Suppose you can arrange things so that they are not matched. Then you have a mermaid, $X$, and a merman, $Y$, that give us the blocking pair $(Y, Ariel)$ and $(Eric, X)$ since $Ariel$ prefers $Eric$ over all others and therefore also $Y$, and $Eric$ prefers $Ariel$ over all others and therefore also $X$.

This contradicts the fact that the Gale-Shapley algorithm returns a stable matching, so they must be matched.

\section{Weak/strong stable marriages}
\subsection*{a) Strong instability}
If we break ties arbitrarily and run the Gale-Shapley algorithm we always find a stable matching where there is no strong instability. \\
\subsection*{b) Weak instability}
Any set of men and women that are completely indifferent to the matching will always have weak instabilities.\\
For example a set of two men and two women where both men and both women are indifferent to the other sex.

\section{Topological sort}
L $\leftarrow$ Empty list that will contain the sorted elements\\
S $\leftarrow$ Set of all nodes with no incoming edges\\
\textbf{while} S is non-empty \textbf{do}\\
\indent remove a node n from S\\
\indent add n to tail of L\\
\indent \textbf{for each }node m with an edge e from n to m \textbf{do}\\
\indent \indent remove edge e from the graph\\
\indent \indent \textbf{if} m has no other incoming edges \textbf{then}\\
\indent \indent \indent insert m into S\\
\textbf{if} graph has edges \textbf{then}\\
\indent \textbf{return} error (graph has at least one cycle)\\
\textbf{else} \\
\indent return L (a topologically sorted order)\\

There is always at least one node in S in the beginning since the graph is a DAG and
S is only empty when all the nodes have been moved to L. \\
While S is not empty each iteration of the while loop moves a node from S to L and L keeps a topological order since only nodes with no incoming edges are added to L and therefore a node can never be added before it's predecessor.\\
Finding the initial S is O(n), the while loop removes a node in every loop so it is goes n loops, and the for each loop goes in total once for each edge so it goes m loops in total. That means that the algorithm is O(n+m).

\subsection{Alternative approach}
As the hint suggests, we will describe an algorithm that uses a depth-first search. We start by describing this DFS algorithm:\\

\begin{algorithm}[H]
  \KwData{Graph $G(V,E)$}
  \KwResult{Order in which vertices were marked, in form of list $L$}
  Initialize all vertices unmarked and an empty stack $S$\;
  Initialize empty list $L$\;
  \While{an unmarked vertex $v$ exists}{
    set $v$ as visited\;
    $u$ = $v$\;
    \Do{$u \not = \emptyset$}{
      \eIf{$u$ has an unmarked and unvisited neighbor $n$}{
        mark $n$ as visited\;
        push $u$ to $S$
        $u$ = $n$
      }{
        set $u$ as marked and append it to the tail of $L$\;
        pop from $S$ and set it to $u$
      }
    }
  }
\end{algorithm}
The result of running the above DFS algorithm is a list with the order that the vertices were marked in. An algorithm for computing a topological sort of $G$, would then be to simply reverse the resulting list from running DFS on $G$:\\

\begin{algorithm}[H]
  \KwData{Graph $G(V,E)$}
  \KwResult{Order in which vertices were marked, in form of list $L$}
  Initialize empty list $L$\;
  $L$ = DFS($G$)\;
  $L$ = reverse($L$)
\end{algorithm}

The running time of this algorithm would then be $O(n+m)$ since it visits every edge and vertex atmost once and reversing the list can be done in linear time.\\

In order to prove the correctness of the algorithm, it should be suficient to prove that if there is and edge $(u,v)\in E$, then $u$ is only marked after $v$. Take the case where the DFS algorithm considers the edge $(u,v)$. At this point $v$ cannot be set to visited, since this would mean that $v$ is an ancestor to $u$, which means that it is either marked or unmarked. If $v$ is marked, then $u$ will be marked after $v$ for sure, affirming the topological sorted order, and if $v$ is unmarked,
it will at some point be popped before $u$ from the stack, and therefore also marked before, again affirming the topological sorted order we wanted to maintain.
\section{Bonus question}

$M_1: anything $ \\
$M_2: W_1 > W_2 > W_3$\\
$M_2: W_2 > W_1 > W_3$\\\\
$W_1: M_3 > M_2 > M_1$\\
$W_2: M_2 > M_3 > M_1$\\
$W_3: anything$\\

0. Starting match: \\
$(M_1, W_1)$, $(M_2, W_2)$, $(M_3, W_3)$ \\
1. Swap for the blocking pair $M_2$ and $W_1$: \\
$(M_1, W_2)$, $(M_2, W_1)$, $(M_3, W_3)$ \\
2. Swap for the blocking pair $M_3$ and $W_1$ \\
$(M_1, W_2)$, $(M_2, W_3)$, $(M_3, W_1)$ \\
3. Swap for the blocking pair $M_3$ and $W_2$ \\
$(M_1, W_1)$, $(M_2, W_3)$, $(M_3, W_2)$ \\
4. Swap for the blocking pair $M_2$ and $W_2$ \\
$(M_1, W_1)$, $(M_2, W_2)$, $(M_3, W_3)$ \\
\\
We get the starting match so this is a loop that could go on forever.
\end{document}

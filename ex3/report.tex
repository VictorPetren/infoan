\documentclass[a4paper]{article}
\usepackage[utf8]{inputenc}
\usepackage{amsmath}
\usepackage{amssymb}
\usepackage{mathtools}
\usepackage{amsfonts}
\usepackage{lastpage}
\usepackage{pdfpages}
\usepackage{fancyvrb}
\usepackage[table]{colortbl}
\usepackage{fancyhdr}
\usepackage[linesnumbered, ruled]{algorithm2e}
\usepackage{graphicx}
\SetKwRepeat{Do}{do}{while}%
\usepackage[margin=2.5 cm]{geometry}


\pagestyle{fancy}
\cfoot{Page \thepage\ of \pageref{LastPage}}
\DeclareGraphicsExtensions{.pdf,.png,.jpg}
\author{Sigurdur Oli Arnason (5961181) \\ Victor Petren Bach Hansen (5990025)}
\title{Algorithms \& Networks \\ Exercise 3}
\lhead{Algorithms \& Networks}
\rhead{Exercise 2}

\begin{document}
\maketitle
\section{Lift Scheduling}

\subsection*{i)}
In the original TSP it doesn't matter how long the salesman stays in each town. Here the task at hand in each town is moving the robot between two floors and optimizing how to move between these tasks is the traveling problem. We define the distance between two tasks, i and j, as the time it takes to go from the final destination of i to the original position of j. Like the salesman can start in his hometown we start in a state which we simulate as a task with final destination 1. In this problem we don't care about going back to floor 1 after completing all the tasks so we define the distance from any task to the starting one as 0. \\
We create a complete digraph with vertices $a_1, a_2, ..., a_k$, one for each AGV, and one vertex $a_0$ representing the starting position with $f_o(a_0) = f_d(a_0) = 1$. The distances are formally defined like this: \\
\[
d(a_i, a_j) = 0 \textrm{ if } f_d(a_i) = f_o(a_j) \textrm{ or } j = 0
\]
\[
d(a_i, a_j) = t_0 + t_1|f_d(a_i) - f_o(a_j)| \textrm{ otherwise }
\]
This becomes an aTSP with k+1 vertices which can be solved with Held-Karp in $\mathcal{O}(2^{k+1}(k+1)^2)=\mathcal{O}(2^{k}k^2)$

\subsection*{ii)}
Build queues for the floors and define $Q\setminus A$ for a queue Q and set A as the queue Q without the elements in A. \\
Change line 3 on slide 29 in Introduction to Exact Exponential-Time Algorithms slides to:\\
If $|S|>1$ then for all $v \neq s$, v is first in $queue(f_o(v))\setminus(S \setminus \{v\})$ \\
Then, when extending the path to the next task, only those tasks that are first in their respective queue, are considered. \\
The added calculation is $\mathcal{O}(k^2)$ so the whole algorithm is $\mathcal{O}(2^{k}k^4)$
\subsection*{iii)}
while(some floor is blocked)\\
\indent flag = false\\
\indent for each(blocked floor f)\\
\indent \indent if (first on f going to non-blocked floor) \\
\indent\indent\indent f.pop \\
\indent\indent\indent flag = true \\
\indent\indent\indent break \\
\indent\indent endif\\
\indent endfor\\
\indent if ( flag == false ) \\
\indent\indent return false \\
\indent endif\\
endwhile \\
return true\\
\\
The while-loop either moves one task from a queue or we have no solution. We have at most k+1 tasks in the queues in total so after the loop runs at most k+1 times no floor can be blocked. The while-evaluation costs $\mathcal{O}(r)$ and the for-loop runs at most r times where each loop costs $\mathcal{O}(1)$. Therefore the whole algorithm runs in $\mathcal{O}(kr)$ which is polynomial.

\subsection*{iv)}
Define function B(f,Q) which returns true if floor f is blocked when exactly the AGVs in queue Q are on the floor, and false otherwise. We change the line from $ii)$ to:\\
If $|S|>1$ then for all $v \neq s$, v is first in $queue(f_o(v))\setminus(S \setminus \{v\})$ and $ \lnot B(f_o(v), queue(f_o(v))\setminus(S \setminus \{v\})$ \\
That is, the AGV is not only first in its queue, but the floor is also not blocked by the queue.
\subsection*{v)}

\section{Maximum Flow by Scaling}
\subsection*{1)}
We know that the upper limit of the number of edges in a cut is $|E|$ edges. We also know that the maximum capacity of any edge in $G$ is $C$, therefore the minimum capacity of any cut is at most $C|E|$.
\subsection*{2)}
Having stated that we need to find an augmenting path in the residual network, where the minimum capacity is $K$, we know that we can simply disregard all edges $(u,v)\in G_f$ where $c(u,v)<K$, as the capacity of a path is equal to the minimum capacity of any edge in the path.

Using the Ford-Fulkerson method (bredth first search or depth first search) of finding an augmenting path in a residual network, we get that we can find it in $\mathcal{O}(|E|)$ time.
\subsection*{3)}
When initialized, $K$ is a power of $2$ and every time the outer while-loop executes $K$ is divided by $2$ and thus remains a power of $2$. In the last iteration of the while-loop, we will have that $K=2^0=1$, which means that the inner while-loop will find all augmenting paths in the residual network with capacity atleast $1$. Since we are assuming integer capacities for the edges, we will in the last iteration find all the remaining augmenting paths, augment flow along them and terminate.

From the max-flow min-cut theorem (CLRS, theorem 26.6), we know that since there are no augmenting paths left after the last iteration, the resulting flow, must be a max-flow.
\subsection*{4)}
When the fourth line is executed (for $K\geq 1$), we know that there does not exists augmenting paths with capacity higher than or equal to $2K$. To see this, when $K$ is initialized, we have that $2K= 2\cdot 2^{\lfloor \lg C \rfloor} > 2\cdot 2^{\lg C -1} = C$, and for each consecutive iteration, we augment all paths $p$, where $K\leq capacity(p)<2K$, which maintains the invariant. Since the capacity of an augmenting path given by the edge along the path with minimum capacity and that
the number of edges is bounded by $|E|$, we have that a minimum cut, will have atmost capacity $2K|E|$.
\subsection*{5)}
We know from 4) that the maximum flow can not increase by more than a total of $2K|E|$ when executing the inner while-loop. We also know that the amount of flow is increased by at least $K$ each time line 6 is executed. This means that the total amount of times that the loop can run is $2K|E|/K=\mathcal{O}(|E|)$ times.
\subsection*{6)}
The inner loop can, by 2), be executed in $\mathcal{O}(|E|)$ time, which, by 5), can execute up to at most $\mathcal{O}(|E|)$ times, yielding a total of $\mathcal{O}(|E|^2)$ time. To see that the outer while-loop runs in $\mathcal{O}(\lg C)$ time, we note that $K = 2^{\lfloor \lg C \rfloor} \leq 2^{\lg C}=C$, which means that $\lg K \leq \lg C$. Since we are unable to divide $K$ by $2$ more than $\lg K$ times, before $K\leq 1$, then the outer loop must be executed at most
$\mathcal{O}(\lg C)$ times, giving us a total run time of $\mathcal{O}(\lg C \cdot |E|^2)$.
\section{Bonus/Research Question: NP-Completeness of lift scheduling}

\end{document}

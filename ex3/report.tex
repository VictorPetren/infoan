\documentclass[a4paper]{article}
\usepackage[utf8]{inputenc}
\usepackage{amsmath}
\usepackage{amssymb}
\usepackage{mathtools}
\usepackage{amsfonts}
\usepackage{lastpage}
\usepackage{pdfpages}
\usepackage{fancyvrb}
\usepackage[table]{colortbl}
\usepackage{fancyhdr}
\usepackage[linesnumbered, ruled]{algorithm2e}
\usepackage{graphicx}
\SetKwRepeat{Do}{do}{while}%
\usepackage[margin=2.5 cm]{geometry}


\pagestyle{fancy}
\cfoot{Page \thepage\ of \pageref{LastPage}}
\DeclareGraphicsExtensions{.pdf,.png,.jpg}
\author{Sigurdur Oli Arnason (5961181) \\ Victor Petren Bach Hansen (5990025)}
\title{Algorithms \& Networks \\ Exercise 2}
\lhead{Algorithms \& Networks}
\rhead{Exercise 2}

\begin{document}
\maketitle
\section{Lift Scheduling}

\subsection*{i)}
Create a strongly connected digraph with vertices $a_1, a_2, ..., a_k$, one for each AGV, and one vertex $a_0$ representing the starting position with $f_o(a_0) = f_d(a_0) = 1$. Define distances as such: \\
$d(a_i, a_j) = 0$ if $f_d(a_i) = f_o(a_j)$ or $j = 0$ \\
$d(a_i, a_j) = t_0 + t_1|f_d(a_i) - f_o(a_j)|$ otherwise \\

\subsection*{ii)}
Build queues for the floors with an operator to remove a set of vertices from a queue.\\
Change slide 29 in Introduction to Exact Exponential-Time Algorithms slides to:\\
If $|S|>1$ then for all $v \neq s$, v is first in $queue(f_o(v))\setminus(S \setminus \{v\})$
\subsection*{iii)}
while(some floor is blocked)\\
\indent flag = false\\
\indent for each(blocked floor f)\\
\indent \indent if (first on f going to non-blocked floor) \\
\indent\indent\indent f.pop \\
\indent\indent\indent flag = true \\
\indent\indent\indent break \\
\indent\indent endif\\
\indent endfor\\
\indent if ( flag == false ) \\
\indent\indent return false \\
\indent endif\\
endwhile \\
return true
\subsection*{iv)}
Define operation blocked(f,Q) which returns true if floor f where exactly the AGVs in queue Q are on the floor is blocked, and false otherwise. \\
If $|S|>1$ then for all $v \neq s$, v is first in $queue(f_o(v))\setminus(S \setminus \{v\})$ and $ \lnot blocked(f_o(v), queue(f_o(v))\setminus(S \setminus \{v\})$
\subsection*{v)}

\section{Maximum Flow by Scaling}
\subsection*{1)}
We know that the upper limit of the number of edges in a cut is $|E|$ edges. We also know that the maximum capacity of any edge in $G$ is $C$, therefore the minimum capacity of any cut is at most $C|E|$.
\subsection*{2)}
Having stated that we need to find an augmenting path in the residual network, where the minimum capacity is $K$, we know that we can simply disregard all edges $(u,v)\in G_f$ where $c(u,v)<K$, as the capacity of a path is equal to the minimum capacity of any edge in the path.

Using the Ford-Fulkerson method (bredth first search or depth first search) of finding an augmenting path in a residual network, we get that we can find it in $\mathcal{O}(|E|)$ time.
\subsection*{3)}
When initialized, $K$ is a power of $2$ and every time the outer while-loop executes $K$ is divided by $2$ and thus remains a power of $2$. In the last iteration of the while-loop, we will have that $K=2^0=1$, which means that the inner while-loop will find all augmenting paths in the residual network with capacity atleast $1$. Since we are assuming integer capacities for the edges, we will in the last iteration find all the remaining augmenting paths, augment flow along them and terminate.

From the max-flow min-cut theorem (CLRS, theorem 26.6), we know that since there are no augmenting paths left after the last iteration, the resulting flow, must be a max-flow.
\subsection*{4)}
When the fourth line is executed (for $K\geq 1$), we know that there does not exists augmenting paths with capacity higher than or equal to $2K$. To see this, when $K$ is initialized, we have that $2K= 2\cdot 2^{\lfloor \lg C \rfloor} > 2\cdot 2^{\lg C -1} = C$, and for each consecutive iteration, we augment all paths $p$, where $K\leq capacity(p)<2K$, which maintains the invariant. Since the capacity of an augmenting path given by the edge along the path with minimum capacity and that
the number of edges is bounded by $|E|$, we have that a minimum cut, will have atmost capacity $2K|E|$.
\subsection*{5)}
We know from 4) that the maximum flow can not increase by more than a total of $2K|E|$ when executing the inner while-loop. We also know that the amount of flow is increased by at least $K$ each time line 6 is executed. This means that the total amount of times that the loop can run is $2K|E|/K=\mathcal{O}(|E|)$ times.
\subsection*{6)}
The inner loop can, by 2), be executed in $\mathcal{O}(|E|)$ time, which, by 5), can execute up to at most $\mathcal{O}(|E|)$ times, yielding a total of $\mathcal{O}(|E|^2)$ time. To see that the outer while-loop runs in $\mathcal{O}(\lg C)$ time, we note that $K = 2^{\lfloor \lg C \rfloor} \leq 2^{\lg C}=C$, which means that $\lg K \leq \lg C$. Since we are unable to divide $K$ by $2$ more than $\lg K$ times, before $K\leq 1$, then the outer loop must be executed at most
$\mathcal{O}(\lg C)$ times, giving us a total run time of $\mathcal{O}(\lg C \cdot |E|^2)$.
\section{Bonus/Research Question: NP-Completeness of lift scheduling}

\end{document}
